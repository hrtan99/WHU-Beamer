\section{一些样式}


\subsection{定理块}

\begin{frame}
    % 定理环境需要两个必须参数 二者可以为空{} 第一个参数是定理块的名称 第二个参数用于引用
    \begin{defn}{Definition}{}
    Some definition.
    \end{defn}
    Here is some description.
    \begin{theo}{Theorem}{}
    Some Theorem.
    \end{theo}
    \begin{theo}{Theorem}{}
    Some Theorem.
    \end{theo}
    
\end{frame}


\begin{frame}
    \begin{coro}{Collary}{}
        Some Collary.
    \end{coro}

\end{frame}

\begin{frame}
    % For multi-column typesetting of listings the listings package provides the multicols=n option, which in fact is a built-in interface to the multicol package.
    \lstinputlisting[caption=Scheduler,style=customc,multicols=2]{codes/hello.c}
\end{frame}



